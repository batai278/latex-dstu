\chapterConclusion

В даному розділі було досліджено відомі статистичні методи обробки даних
та психологічні особливості студентів, що домопожуть створити систему
тестування, яка буде більш об’єктивною, ніж класичні аналоги.
Також було розпочато моделювання, за допомогою якого будуть виконані
перші кроки в створенні та тестуванні системи.

Метод головних компонент буде використано для виявлення найвагоміших
рис (головних компонент) поведінки студентів під час виконання тестів.
Для його застосування було вирішено розбити час розв’язання тестових завдань
на скінченну кількість відрізків --- дискретизувати час.

За допомогою критерію узгодженості Пірсона буде перевірено гіпотези щодо
розподілів головних компонент, які були знайдені на попередньому кроці.
Оскільки вони будуть лінійними комбінаціями інших елементів виборки, можна
припустити, що розподіл буде нормальним --- залишиться визначити середнє
та дисперсію кожної головної компоненти.

Дані з психології надали матеріал, на основі якого можна розділити людей
на кілька категорій, кожна з яких має свої особливості.
На основі цього було вирішено, які поради система буде надавати викладачеві
практичних занять щодо роботи з різними студентами.
