\chapter*{Висновки}
\addcontentsline{toc}{chapter}{Висновки}

В результаті виконання роботи вдалося побудувати математичну модель
психологічних характеристик людини на основі результатів тестування.

Було використано теппінг-тест, генерацію процесу Пуассона та критерій Пірсона
для побудови моделі психологічних особливостей людини.
Інформаційний метод прогнозування часу розв’язання завдань та генерація
випадкових величин, що розподілені за експоненційним законом, були використані
для побудови моделі проходження тестів змодельованими студентами.
За допомогою методу головних компонент та дерева класифікації було
проаналізовано та класифіковано результати проходження тестів.

Створена модель може бути розвинута та застосована на практиці для надання порад
викладачам практичних занять щодо роботи з конкретними студентами.
Про це сідчить те, що в ході дослідження було знайдено джерела з інженерної
психології та психофізіології, де вказано, що подібні дослідження виконувалися.
Оскільки детальних досліджень в галузі освіти не було знайдено, робота є
актуальною і має потенціал розвитку.
