\chapter*{Реферат}

Неотъемлемой частью оценки успеваемости студентов является анализ результатов
тестирования, однако существующие на данный момент системы тестирования
учитывают лишь правильность ответов.

Цель данной работы --- разработка математической модели психологических
особенностей человека на основе результатов тестирования.
Это поможет создать тестирующую систему, способную давать нужные советы
людям, которые прошли тестирование.

Для достижения поставленной цели были использованы
\begin{itemize}
  \item 
    главных компонент для извлечения наиболее значимых  данных из анализируемых
    выборок;
  \item
    критерий Пирсона ($\chi^2$) для проверки гипотез о распределении выборок;
  \item
    данные из психофизиологии для построения модели психологических
    характеристик людей для дальнейшего использования в моделировании экзамена;
  \item
    данные из инженерной психологии для построения модели прохождения экзамена.
\end{itemize}
