\chapter*{Реферат}

Аналіз результатів тестування є невід’ємною частиною перевірки рівня знань
студентів, проте існуючі на даний момент системи тестування враховують
лише вірність відповідей.

Метою даної роботи є побудова математичної моделi психологiчних характеристик
людини на основi результатiв тестування.
Це допоможе створити тестуючу систему, що може надавати доречні поради людям,
що пройшли тестування.

Для досягнення мети
\begin{itemize}
  \item 
    було використано метод головних компонент,
    щоб отримати найбільш важливі дані з аналізованих вибірок;
  \item
    було використано критерій Пірсона ($\chi^2$) для перевірки гіпотез щодо
    розподілу вибірок.
\end{itemize}

