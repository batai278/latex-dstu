\chapter*{Реферат}

Аналіз результатів тестування є невід’ємною частиною перевірки рівня знань
студентів, проте існуючі на даний момент системи тестування враховують
лише вірність відповідей.

Метою даної роботи є побудова математичної моделi психологiчних характеристик
людини на основi результатiв тестування.
Це допоможе створити тестуючу систему, що може надавати доречні поради людям,
що пройшли тестування.

Для досягнення мети було використано
\begin{itemize}
  \item 
    метод головних компонент, щоб отримати найбільш значущі дані з аналізованих
    вибірок;
  \item
    критерій Пірсона ($\chi^2$) для перевірки гіпотез щодо розподілу виборок;
  \item
    дані з психофізіології для побудови моделі психологічних характеристик
    людей для подальшого використання в моделюванні екзамену;
  \item
    дані з інженерної психології для побудови моделі проходження екзамену.
\end{itemize}

\MakeUppercase{Метод головних компонент, критерій Пірсона, теппінг-тест,
психофізіологія, інженерна психологія}
