\chapter{Охорона праці}

Робота відбувається у робочому приміщенні.
У цьому приміщенні розроблялася математична модель психологічних особливостей
людини на основі результатів тестування.
В приміщенні розташовано один комп’ютер класу ноутбук Fujitsu LH532,
Intel Core i3-2328M 2.2GHz, nVidia GeForce GT 620M,
жорсткий диск 500 ГБ з монітором 1366x768 типу HD LED.

\section{Технічні рішення щодо безпечного виконання роботи}

В приміщенні працює одна людина, яка працює за описаним вище ноутбуком.
Параметри кімнати:
довжина --- $7$ метрів, ширина --- $5$ метрів, висота --- $3$ метри,
площа --- $35$ метрів квадратних, об’єм $105$ --- метрів кубічних.

Оскільки у кімнаті працює одна людина, то на неї відводиться уся площа
35 метрів квадратних і об’єм $105$ метрів кубічних.
Згідно \cite{NPAOP} і \cite{DSanPiN} обсяг простору для однієї людини
в лабораторії з ЕОМ повинен бути не менше $20$ метрів кубічних,
а площа --- не менше $6$ метрів квадратних.
Оскільки фактичні параметри перевищують нормативні,
то приміщення задовольняє перерахованим вимогам.

Робочі місця знаходяться не менш ніж на один метр від стін з вікнами.
Габарити робочого місця операторів ЕОМ:
висота стола --- 800 мм, ширина --- 2500 мм, глибина --- 1200 мм.
Простір для ніг складає: висота --- 650 мм, ширина --- 1000 мм,
глибина на рівні колін --- 550 мм, а на рівні простягнутої ноги --- 1200 мм.
Робоче крісло оператора належить до підйомно-поворотного типу,
з регулюванням висоти та повороту спинки.
Висота спинки та її ширина мають розміри 750 мм та 550 мм відповідно.
Крісло також обладнане стаціонарними підлокітниками довжиною 260 мм
та шириною 70 мм.

Екран монітору розташовано на відстані 550 мм від очей оператора,
а габарити робочого місця дозволять за необхідності розмістити його ще далі.
Конструктивні особливості клавіатури передбачають її вільне переміщення
з можливістю регулювання кута нахилу.

\section{Електробезпека}

Робоче приміщення обладнано стандартною трипровідною однофазною електричною
мережею змінного струму з напругою 220 В.

Через те, що температура час від часу підвищується до $26-27$C,
а вiдносна вологiсть знаходиться на рівні $40-50\%$,
приміщення відноситься до групи приміщень без підвищеної небезпеки,
адже температура нижче $30$C, а вологість не перевищує $60\%$.

Споживачі електроенергії в приміщенні --- освітлювальні прилади
і обчислювальна техніка (ноутбук).

Всі прилади, що працюють від електромережі, занулені.
Електрична щитова знаходиться в кімнаті,
що полегшую вимикання струму в разі нагальної потреби.
Корпуси сучасних ноутбуків виготовлені із пластмаси,
тому ураження струмом неможливе.

Уся електропроводка та мережеві кабелі знаходяться в закритому положенні
у спеціальних пластикових закритих коробках.
Провід марки ВВП $3\times 1.5$ з мідними струмопровідними жилами з ПВХ пластика;
плоский, з розділювальною основою.
Номінальна змінна напруга $220$В, частота --- до $50$ Гц.

Запобіжна система на трансформаторі спрацьовує при силі струму в $27$ А.
Усі параметри не перевищують граничних меж.

\section{Пожежна безпека}

Конструкція будівлі, де знаходиться приміщення, виконана із залізобетонних плит,
тобто його конструктивні елементи не спалювані.
Будівля відноситься до II ступеня вогнестійкості.
Межа вогнестійкості конструкції приблизно $0.5 - 2.5$ год.

Згідно до \cite{ONTP} по вибухонебезпечній і пожежній небезпеці приміщення
відноситься до категорії В, тому що в ньому перебувають важкогорючі тверді
й волокнисті речовини й матеріали: стіл із дерева, ноутбук, периферія, лінолеум
та ін.
Оскільки в приміщенні знаходяться тверді горючі речовини та матеріали,
немає горючих рідин і не накопичуються горючий пил,
за класифікацією воно належить до класу П-IІа.

Для підвищення безпеки, зокрема і пожежної, мережа напругою з $220$ В
є проводом з мідною жилою.
Розташовується у вінілопластикових трубах, що прокладені в стінах.
Ізоляція проводів розрахована на напругу в $1.5$ кВ. 

В приміщенні встановлена пожежна сигналізація СД121-5, оповісник ІПК-1,
комбінований з пульсуючою індикацією, що реагує на задимленість та температуру.
Сигналізація приміщення підключена до централізованого інформаційного пункту,
від якого у випадку виникнення пожежі сигнал передається на пожежну частину.
Крім цього біля виходу з приміщення знаходиться вогнегасник типу ОУ-2.
Це не відповідає вимогам пожежної безпеки згідно до \cite{PPBU},
адже потрібно не менше двох вогнегасників на 20 квадратних метрів,
отже на дане приміщення площею 35 квадратних метри потрібно як мінімум
чотири вогнегасники типу ОУ-2.

Для приміщення, в якому працюють менш ніж $25$ людей, та відстань від будь-якого
робочого місця не перевищує $25$ метрів згідно норм досить одного виходу
евакуації. Параметри евакуаційного виходу відповідають встановленим нормам.
Двері відкриваються назовні.
При нормі не менше $2$ м ширина коридору $2.2$ м.
Висота до перекриття $3$ м при нормі не менше $3$ м.
Висота дверей у коридорі $2$ м, а ширина $1.8$ м, що також відповідає нормі.
Ширина дверей рівна $0.8$ м  (норма --- $0.8$ м).

\section{Виробничий шум}

Приміщення знаходиться в будівлі, яка перебуває на значній відстані від вулиці;
гучних об’єктів поблизу немає.
Також у приміщенні встановлені звукоізолюючі металопластикові вікна,
а система кондиціонуванняспроектована таким чином,
що гучні її елементи знаходяться на інших поверхах будівлі.
Отже, джерелами шуму є лише комп’ютерна техніка, а точніше деякі її складові.

За результатами вимірювання звукового шуму на робочому місці,
приладом Шумомір ВШВ-003-М2, рівень шуму коливався в межах від 25 до 35 дБА, 
що не перевищує поріг у 50 дБА та задовольняє санітарним нормам \cite{DSanPiN}.

\section{Виробниче освітлення}

При незадовiльному освiтленнi зменшується продуктивнiсть працi
користувачiв персональних комп’ютерiв.

Тип природного освітлення: бокове одностороннє.
Згідно \cite{DBN} для категорії робіт дуже високої точності(
найменший розмір об’єкту розрізнення $0.15-0.3$ мм,
а розмір пікселя дисплею --- $0.263$ мм)
нормоване значення КПО $e_\text{н} = 1.5$.
Враховуючи географічне розміщення приміщення,
коефіцієнт світлового клімату $m = 0.9$.
Тоді нормативне значення КПО: 
\begin{equation*}
e_{N} = 1.5 \times 0.9 = 1.35.
\end{equation*}
Розрахуємо фактичне значення КПО.
Фактичне значення КПО визначається за формулою: 
\begin{equation*}
e_\textbf{ф}=\dfrac{100S_0\tau r}{S_h \eta K_\text{з}K_\text{зд}}a,
\end{equation*}
 де
$S_0$ --- площа поверхонь всіх вікон в приміщенні;
$S_h$ --- площа підлоги в приміщенні м$^2$;
$\tau$ --- загальний коефіцієнт світлопроникності приміщення,
який для віконних прорізів громадських приміщень,
що не мають сонцезахисного пристрою, рівний $0.5$;
$r$ --- коефіцієнт, що враховує відбиття світла від внутрішніх поверхонь;
$\eta$ --- світлова характеристика вікна; 
$K_\text{з}$ --- коефіцієнт запасу, береться у межах $1.3$-$1.5$;
$K_\text{зд}$ --- коефіцієнт, що враховує затемнення вікон іншими будівлями,
який дорівнює $1$, якщо будівель немає.

\begin{equation*}
  e_\phi
  =\dfrac{100 \cdot 3.15 \cdot 0.5 \cdot 1.05}{35 \cdot 11 \cdot 1.3}
  = 0.33
\end{equation*} 

При порівнянні фактичного значення освітленості з нормованим видно,
що природне освітлення не є ефективним.

У ролі штучного освітлення виступають $3$ світильника з сумарно кількістю в
$9$ люмінесцентних ламп ЛБ 36, згідно технічних характеристик даного типу ламп,
світловий потік рівний 3050 лм.
Кожен світильник містить по три  лампи. 
Розрахуємо фактичну освітленість робочого приміщення для порівняння з нормою,
враховуючи відстань кожного світильника та кількості ламп у ньому:
\begin{equation*}
  E_{\text{н}} = I/R^{2} \times \cos\alpha = F/(4 \times \pi \times R^{2} ),
\end{equation*}
\begin{equation*}
  E_{1} = 3050/(4\times 3.14\times 4)\times \cos{45^{\circ}} = 42.7 \text{лк},
\end{equation*}
\begin{equation*}
  E_{2} = 3050/(4\times 3.14\times 6.25)\times \cos{60^{\circ}} = 32.7\text{лк},
\end{equation*}
\begin{equation*}
  E_{3} = 3050/(4\times 3.14\times 2.25)\times \cos{75^{\circ}} = 104 \text{лк},
\end{equation*}
де $I$ - сила світла, $R$ - відстань до джерела світла, 
$\Phi$ - світловий потік, $\alpha$ - кут падіння променя світла.

Сумарне значення буде рівним
\begin{equation*}
  E_1 \times 3 + E_2 \times 3 + E_3 \times 3
  = 42.7 \times 3 + 32.7 \times 3 + 104 \times 3
  = 538.2 \text{лк}
\end{equation*}

Для даного типу роботи, згідно \cite{DBN}, нормативне значення освітленості
дорівнює $E_{\text{н}} = 500$лк. 
Оскільки $E_{\text{н}} < E$, то штучне освітлення робочого приміщення
відповідає санітарним нормам.

\section{Мікроклімат}

Якiсть повiтря в примiщеннi впливає на самопочуття людини.
У примiщеннях з комп’ютерною технiкою, внаслiдок того що третина споживаної нею
енергiї розсiюється у виглядi тепла, формуються специфiчнi умови мiкроклiмату
а саме, пiдвищується температура до $26-27$ C,
вiдносна вологiсть падає до $40\%$.

У виробничих примiщеннях на робочих мiсцях з персональним комп’ютером мають
забезпечуватись оптимальнi значення параметрiв мiкроклiмату: температури,
вiдносної вологостi й швидкостi руху повiтря. \cite{DSanPiN}

Розробка дипломного проекту вiдноситься до категорiї легких --- 1A робiт.

В холодний період температура тримається на рівні 21-23 (норма --- 22-24),
вологість $50$-$55$\% (норма --- $40$-$60$\%), а швидкість руху повітря
$0.04-0.08$ м/с (норма --- $<0.1$ м/с).
В теплий період температура тримається на рівні 22-25 (норма --- 23-25),
вологість $55$-$60$\% (норма --- $40$-$60$\%), а швидкість руху повітря
$0.03-0.08$ м/с (норма --- $<0.1$ м/с).
Для забезпечення постiйного параметру мiкроклiмату в примiщеннi встановлений
побутовий кондицiонер, що вмикається за необхiднiстю.
Отже, мікроклімат приміщення задовiльняє норми наведенi у \cite{DSN}.

\section{Організація оптимального режиму праці та відпочинку}

При роботі з ПК для підтримання продуктивної діяльності робітника та запобіганню
професійних захворювань передбачаються регламентовані перерви для відпочинку
у спеціально відведених для цього місцях.
До таких місць належить гімнастичний зал, де можна виконувати вправи
для підтримання тонусу опорно-рухової системи, 
адже в основному робота несе сидячий характер.
При виконанні протягом робочої зміни робіт, що відносяться до різних видів
трудової діяльності, за основну роботу з засобами обчислювальної техніки
слід приймати таку, яка займає не менше $50\%$ часу протягом робочого дня.
Для забезпечення високого рівня праці організаційно передбачено:
\begin{enumerate}
\item перерви для  вживання їжі (обідні перерви);
\item перерви для відпочинку й особистих потреб;
\item проведення вправ для очей через кожні 20-25 хвилин роботи;
\item проведення під час перерв наскрізного провітрювання приміщень;
\item здійснення під час перерв вправ фізкультурної паузи протягом 15-20 хвилин.
\item Перерви для відпочинку через кожну годину тривалістю 10 хвилин
\end{enumerate}
Для зменшення негативного впливу монотонної роботи з ПК передбачено чергування
операцій осмисленого тексту і числових даних, чергування редагування тексту та
введення даних.
Також, для зменшення  нервового та емоційного рівнів напруження та зняття втоми
з очей організаційно передбачено проведення під час перерв спеціальних
комплексів вправ.

\section{Склад повітря робочої зони}

Основним джерелом забруднення повітря робочої зони є пил. 
Джерелами пилу в приміщенні є: документи, папір, книги, одяг, взуття, зовнішнє
повітря.
За допомогою пиломіру ФПГ-6 було встановлене фактичне значення концентрації
пилу 4мг/м$^3$,що не перевищило значення ГДК для нейтрального пилу,
що не має отруйних властивостей, яке рівне 10 мг/м$^3$. 
Тому негативна дії на людину не здійснюється. 
Задля очищення приміщення від пилу проводиться періодичне вологе
або сухе прибирання та провітрювання приміщення. 
Вуглекислий газ, що утворюється від дихання людей, видаляється вентиляцією.

\section{Виробничі випромінювання}

Джерелом випромінювань під час роботи є монітор ноутбука. 
Вид випромінювань --- електромагнітні (ЕМ), ультрафіолетові, інфрачервоні,
діапазон випромінювань --- $5$Гц–$2$ГГц.

Згідно з \cite{DSN} гранично допустиме значення електромагнітного поля
комп’ютера $25$В/м.
Монітор, що використовується на робочому місці, має електромагнітне поле
не більше $1$В/м.
Для вимірювання використано вимірювач напруженості електричного
і магнітного поля промислової частоти ($50$ Гц) П3-50.

Дані вимоги дотримані на робочому місці і відповідають нормативним документам.

\chapterConclusion

У даному розділі досліджені питання охорони праці при роботі з візуальними
дисплейними терміналами ЕОМ.
Були виявлені й вивчені небезпечні й шкідливі виробничі фактори.
Взагалом було виявлено, що
обсяг і площа приміщення  відповідають нормативним значенням,
параметри робочого місця задовольняють нормам,
показники мікроклімату цілком відповідають нормативним вимогам,
відповідність нормам по природньому освітленні, наявність штучного освітленням,
акустичні умови роботи в нормі,
при дотриманні правил електробезпечності виключається поразки персоналу
електричним струмом,
але приміщення не повністю задовольняє нормам пожежної безпеки.
