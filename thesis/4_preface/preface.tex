\chapter*{Вступ}
\addcontentsline{toc}{chapter}{Вступ}

\textbf{Актуальність роботи} полягає в тому, що
дистанційне навчання стає дедалі популярнішим, а
існуючі на даний момент системи тестування недостатньо гнучкі: вони
аналізують лише відповіді на запитання, відносячи їх до вірних або невірних,
а на цій базі роблять кінцевий висновок щодо знань студента.
З одного боку, це підвищує їх об’єктивність, з іншого, навпаки: всі студенти
різні і більшості з них потрібна допомога хоча б у вигляді порад.
У викладачів може не вистачати часу на докладне знайомство з кожним студентом,
але якщо буде програмно-технічний комплекс, що підкаже викладачам, до яких
студентів який підхід краще мати, це безумовно підвищить якість навчання.
Стрімкий розвиток комп’ютерної техніки й інформаційних технологій надає
можливість визначати ритм складання тесту, а також індивідуальні особливості
людини.
Дані психологічних досліджень допоможуть правильно трактувати отримані
значення, а добре вивчені та перевірені часом математичні методи надають
великі можливості для систематизації та обробки результатів вимірювання.

\textit{Об’єкти дослідження}:
студенти, системи тестування.

\textit{Предмети дослідження}:
психологічні особливості студентів, моделі їх поведінки.

\textit{Задача}:
Побудувати математичну модель психологічних характеристик людини на основі
результатів тестування.

\textit{Метою роботи є}
покращення якості навчання за допомогою порад студентам і викладачам
практичних занять.

Завдання наступні:
\begin{enumerate}
  \item
    Вивчити математичні методи та розділи психології, що дозволять розробити
    математичну модель психологічних характеристик людини,
    пояснити та обґрунтувати отримані результати
  \item
    Розробити математичну модель людини, що складає тести
  \item
    Розробити та проанализувати модель складання тестів людьми різних типів
\end{enumerate}

