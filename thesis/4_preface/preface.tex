\chapter{Вступ}
\section{Обґрунтування та актуальність роботи}
Існуючі на даний момент системи тестування недостатньо гнучкі: вони
аналізують лише відповіді на запитання, відносячи їх до вірних або невірних,
а на цій базі роблять кінцевий висновок щодо знань студента.
Стрімкий розвиток комп’ютерної техніки й інформаційних технологій надає
можливість визначати ритм складання тесту, а також індивідуальні особливості
людини.
Дані психологічних досліджень допоможуть правильно трактувати отримані
значення, а добре вивчені та перевірені часом математичні методи надають
великі можливості для систематизації та обробки результатів вимірювання.

\section{Мета та завдання}
Завдання наступні:
\begin{enumerate}
  \item
    Вивчити математичні методи та розділи психології, що дозволять розв’язати
    поставлену задачу, пояснити та обґрунтувати отримані результати
  \item
    Ознайомитися з правилами побудови тестових завдань для найбільш
    ефективної та об’єктивної процедури оцінки знань студентів
  \item
    Розробити програмний комплекс тестування й обробки результатів
  \item
    Моделювання
\end{enumerate}

За мету поставлено збільшення об’єктивності тестування, а також покращення
якості навчання за допомогою порад студентам і викладачам практичних занять.
