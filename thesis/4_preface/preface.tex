\chapter*{Вступ}
\addcontentsline{toc}{chapter}{Вступ}

\textbf{Актуальність роботи} полягає в тому, що
дистанційне навчання стає дедалі, а
існуючі на даний момент системи тестування недостатньо гнучкі: вони
аналізують лише відповіді на запитання, відносячи їх до вірних або невірних,
а на цій базі роблять кінцевий висновок щодо знань студента.
З одного боку, це підвищує їх об’єктивність, з іншого, навпаки: всі студенти
різні і більшості з них потрібна допомога хоча б у вигляді порад.
У викладачів може не вистачати часу на докладне знайомство з кожним студентом,
але якщо буде програмно-технічний комплекс, що підкаже викладачам, до яких
студентів який підхід краще мати, це безумовно підвищить якість навчання.
Стрімкий розвиток комп’ютерної техніки й інформаційних технологій надає
можливість визначати ритм складання тесту, а також індивідуальні особливості
людини.
Дані психологічних досліджень допоможуть правильно трактувати отримані
значення, а добре вивчені та перевірені часом математичні методи надають
великі можливості для систематизації та обробки результатів вимірювання.

\textit{Об’єкт дослідження}:
студенти, системи тестування.

\textit{Предмет дослідження}:
психологічні особливості студентів, моделі їх поведінки.

\textit{Метою роботи є}
збільшення об’єктивності тестування, а також покращення
якості навчання за допомогою порад студентам і викладачам практичних занять.

Завдання наступні:
\begin{enumerate}
  \item
    Вивчити математичні методи та розділи психології, що дозволять розв’язати
    поставлену задачу, пояснити та обґрунтувати отримані результати
  \item
    Ознайомитися з правилами побудови тестових завдань для найбільш
    ефективної та об’єктивної процедури оцінки знань студентів
  \item
    Розробити програмний комплекс тестування й обробки результатів
  \item
    Моделювання
\end{enumerate}


