% Preface template:
% \textit{Об’єкт дослідження}:

% \textit{Предмет дослідження}:

% \textit{Метою роботи є}

\textit{Об’єкт дослідження}:
користувачі, експертні системи, їх взаємодія.

\textit{Предмет дослідження}:
поведінка користувачів при навчанні, реакція на різноманітний зовнішній вплив,
та використання отриманої інформації для розробки експертної системи оцінювання
і класифікації перших.
Окремий інтерес являє собою правильність трактування даних, що отримуються від
користувачів в результаті взаємодії з експертною системою, --- важливіша і
складніша частина роботи, що і відрізняє її від аналогів.

\textit{Метою роботи є}
побудова системи комп’ютерного навчання користувачів на прикладі обробки
геоінформаційних даних.
Така задача є достатньо багатогранною, для її втілення потрібно працювати
з даними різного походження, також природньо виникає потреба використання
розподіленої системи.
На такій базі буде розроблено достатньо загальний метод навчання,
що використовуватиметься для інших дисциплін.

Основна задача --- мінімізувати витрати часу експерта (викладача) на процес
перевірки знань користувачів (студентів) та надання їм навчальних матеріалів,
підвищення об’єктивності оцінювання користувачів, а також оцінка роботи
самих викладачів.

Тут постає питання інформаційної безпеки, адже потрібно контролювати
користувачів --- перевіряти, чи та людина сидить за своїм робочим місцем,
запобігати списування, гарантувати максимальну безпеку базі даних з
результатами робіт, а також стежити за тим, щоб користувачі отримували коректні
завдання, а розв’язки відправлялися системі в незміненому вигляді.
