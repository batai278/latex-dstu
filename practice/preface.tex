% Preface template:
% \textit{Об’єкт дослідження}:

% \textit{Предмет дослідження}:

% \textit{Метою роботи є}

\textit{Об’єкт дослідження}:
користувачі, \glslink{expertsystem}{експертні системи}, їх взаємодія.

\textit{Предмет дослідження}:
поведінка користувачів при навчанні, реакція на різноманітний зовнішній вплив.

Окремий інтерес собою являє правильність трактування даних, що отримуються від
користувачів в результаті взаємодії з експертною системою, --- важливіша і
складніша частина роботи, що і відрізняє її від аналогів.

%Також буде розглядатися питання \glslink{ergonomics}{ергономіки} інтерфейсу
системи для найбільш ефективного процесу навчання.

\textit{Метою роботи є}
побудова експертної системи \glslink{computerstudy}{комп’ютерного навчання}
користувачів на прикладі обробки геоінформаційних даних.
Така задача є достатньо багатогранною, для її втілення потрібно працювати
з даними різного походження, також природньо виникає потреба використання
розподіленої системи.
На такій базі буде розроблено достатньо загальний метод навчання,
що використовуватиметься для інших дисциплін.

Основна задача --- мінімізувати витрати часу експерта (викладача) на процес
перевірки знань користувачів (студентів) та надання їм навчальних матеріалів,
підвищення об’єктивності оцінювання користувачів, а також оцінка роботи
самих викладачів.
Це буде зроблено шляхом розробки експертної системи, що буде аналізувати дії
користувачів, класифікувати їх, робити відповідні висновки.

Тут постає питання інформаційної безпеки, адже потрібно контролювати
користувачів --- перевіряти, чи та людина сидить за своїм робочим місцем,
запобігати списування, гарантувати максимальну безпеку базі даних з
результатами робіт, а також стежити за тим, щоб користувачі отримували коректні
завдання, а розв’язки відправлялися системі в незміненому вигляді.
