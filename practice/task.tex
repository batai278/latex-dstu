\section{Власне, завдання}
Як вже було описано вище, індивідуальне завдання --- створення розподіленої
геоінформаційної системи, призначення якої --- навчання обробці і використанню
даних, що фігурують в таких системах.
Це є векторні та растрові дані, таблиці, тексти, числа, діаграми і таке інше
--- тобто, дані різноманітного походження, які потребують застосування різних
методів зберігання та обробки, але треба створити єдину систему, що буде
коректно працювати з усім цим розмаїттям єдиним способом.

Далі ця система буде розширена і узагальнена на інші галузі і інші потреби
--- не тільки навчання студентів, але й навчання та підвищення кваліфікації
персоналу підприємств, і не тільки для використання та обробки геоданих.

\section{Актуальність проблеми}
В світі інформаційних технологій жити стає дедалі зручніше.
Завдяки потужним комп’ютерам і мережі Інтернет стає можливою передача великих
обсягів інформації за відносно малий час.
Чому б не використати можливості сучасних інформаційних технологій для
покращення процесу навчання?

Різні викладачі мають різну методику викладання, різні вимоги до
студентів, по-різному запобігають списуванню.
Як щодо того, щоб об’єднувати знання і навички викладачів, обирати
найефективніші?

Підемо далі! Можна створити самодостатню систему, що зможе самостійно наглядати
за студентами та синтезувати нові правила класифікації користувачів, щоб
викладач міг приділяти більше часу важливим консультаціям, корективам до курсу,
створенню та внесенню змін до практичних завдань, або ж просто на відпочинок.

\section{Аналоги}
Перші кроки вже зроблено --- існують електронні щоденники шкільників,
електронні конспекти з різних дисциплін, електронні тести для перевірки знань.
На мою думку, найбільш розвиненою з таких систем є Coursera \cite{Coursera}.

Coursera надає доступ до навчальних матеріалів (відеолекцій, друкованих
конспектів), проводить перевірку знань у вигляді тестів.
Зокрема проект має дуже цікаву можливість підтвердження особистості за
допомогою веб-камери та електронного почерку, що дозволяє видавати сертифікати
за проходження курсів, яким можна довіряти.
Взаємне оцінювання студентів у спірних питаннях також є продуктивною ідеєю, що
дозволить підвищити об’єктивність оцінки рівня складності завдань та внесе
різноманіття в процес навчання.

Чого ж нам не вистачає?

\subsection{Суб’єктивнe уявлення системи про користувача}
Звісно, студентів ніщо не переможе, але можна почати робити перші кроки в
аналізі поведінки користувачів математичними методами.

Можна будувати емпіричні моделі --- сказати, що цей и цей студент поганий, тому
їх поведінка --- поведінка студентів, що списують.
Або інакше --- створити правило, згідно з яким студент, що погано виконував
практичні завдання, а на екзамені показав гарні результати, обов’язково списав.
Це не є те, що нам потрібно, але від таких спостережень можна відштовхуватись.

Тобто, кінцевий продукт повинен аналізувати дії користувачів, результати
тестування та інші фактори, щоб окрім оцінки правильності виконання робіт
давати ще оцінку чесності отриманих результатів.

\subsection{Вибір завдань для кожного користувача в індивідуальному порядку}
Люди спілкуються один з одним і намагаються одне одному допомогти
(або заважати) --- це природньо.

Розглянемо типічну ситуацію: хтось проходить тест, фотографує завдання, і вже
на наступний раз (на випадок необхідності переписати роботу) у кількох
студентів вже є приклад завдання (а може навіть розв’язаний).

Навіть якщо викладач має багато варіантів завдань, не виключається можливість
того, що один і той самий студент кілька разів отримає один і той самий білет.

Що робити викладачу? Записувати, який студент які завдання виконував?
Дізнаватися, хто з ким товарищує і хто з ким напевно ділиться розв’язками?
На мою думку, це дуже складно і не варто таких зусиль.

Тим не менш, для комп’ютерної системи такі обмеження майже знімаються.
Наприклад, база даних, в якій містяться дані про те, хто, коли і як виконував
певні завдання, а на основі цієї інформації приймати рішення щодо тесту, який
дати певному користувачу. Комп’ютерам можно призначити ідентифікатори, щоб
знати, хто сидить по сусідству з даним користувачем (з приватними комп’ютерами
все складніше, але можна щось запропонувати). Інтеграція системи з соціальними
мережами дасть додаткову інформацію про зв’язки у групах.

Також можуть бути користувачі, що претендують на невисокий бал --- для них
можна намагатися шукати прості завдання, виконання яких не дасть високої
оцінки, але яке вважається достатнім для того, щоб студент продемонстрував свої
знання. Навпаки, для студентів, що подають великі надії, можна обирати
задачі підвищенної складності, що можуть дати додаткові бали.

\section{Прецеденти}

%\subsection{Реальне життя}

Послідовність дій студента за курс навчання якогось предмету можна розкласти
на наступні етапи:
\begin{enumerate}
    \item
        Записатися на курс (вступити до університету)
    \item\label{item:learnTheory}
        Ознайомитися з теорією (слухати лекції)
    \item\label{item:doExercises}
        Виконати практичні завдання для закріплення теорії
        (виконання домашніх завдань, контрольних робіт)
    \item
        Повторювати \ref{item:learnTheory} і \ref{item:doExercises} за
        навчальним планом
    \item
        Продемонструвати свої знання та навички у зазначений строк
        (залік, екзамен, курсова)
\end{enumerate}

Послідовність дій викладача при навчанні:
\begin{enumerate}
    \item
        Стати викладачем
    \item
        Створити план курсу --- план лекцій, практичних занять тощо
    \item
        Ознайомити студентів з курсом (провести лекції)
    \item
        Робити проміжні перевірки знань (контрольні, домашні завдання)
    \item
        Зробити кінцевий контроль знань (екзамен, залік)
\end{enumerate}

З першого погляду схема здається простою, але обов’язково потрібно враховувати
людський фактор --- недобросовісними можуть бути як студенти, так і викладачі.
Студентів багато, а викладач один, тому списування стає справжньою проблемою,
яку різні викладачі вирішують по-різному, але основа одна --- виявлення
основних ознак списування та впровадження штрафних санкцій.

Розглянемо більш детально процес проміжної перевірки знань (написання
контрольних робіт), точніше, огляд схеми, яка запропонована авторами
роботи як еталонна:

\begin{enumerate}
    \item
        Викладач займає своє робоче місце, впускає студентів
    \item
        Кожен студент займає своє робоче місце
    \item
        Кожен студент отримує індивідуальне завдання
    \item
        Викладач (асистент) записує, який студент які завдання отримав
    \item
        Студенти приступають до розв’язання своїх задач
    \item
        Викладач (асистент) слідкує за тим, щоб студенти не списували
    \item
        У разі списування викладач ставить примітку з вказанням ступені,
        оріентованих обсягів списування в журналі, та ліквідує подальшу
        можливість даного студента скористатися джерелом списування --- забирає
        шпаргалку, відсаджує від сусіда тощо
    \item
        Коли студент вирішив, що впорався з завданням як міг, він здає роботу
        і більше не має можливості її редагувати
    \item
        Коли витікає строк складання тесту (контрольної), викладач (асистент)
        збирає те, що виконали студенти, або просить здати роботи самостійно
    \item
        Роботи роздаються студентам на перевірку (свою роботу студент перевіряти
        не може), якщо це передбачено для даної роботи
    \item
        Студенти перевіряють роботи так, щоб можна було відрізнити те, що
        написано хозяїном роботи, від того, що додав перевіряючий. Якщо це не
        виявляється неможливим (наприклад, студент використав усі можливі
        кольори ручок, олівців, тому відрізнити його роботу від приміток
        перевіряючого неможливо), робота одразу потрапляє на перевірку до
        викладача
    \item
        Асистент стежить за чесністю перевірки робіт студентами
    \item
        Коли закінчується строк перевірки, роботи здаються
    \item
        По закінченню заняття студенти залишають приміщення
    \item
        Після студентів приміщення залишає і сам викладач (з асистентом)
    \item
        У вказані строки викладач перевіряє роботи студентів, виставляє оцінки.
        Також викладач може залишати суб’єктивні примітки щодо чесності
        написання роботи і в залежності від них змінювати бали
    \item
        У зазначений час (наприклад, під час практичних занять) студенти повинні
        ознайомитися з результатами
    \item
        Якщо хтось не згоден з результатами, він може попросити проглянути свою
        роботу
    \item
        Викладач (асистент) повинен стежити, щоб студенти не виправляли роботу
        під час проглядання
    \item
        Якщо студент довів (або не довів) свою правоту, оцінка може змінитися.
        Також йому може бути дана робота на переписування (з іншими завнаннями)
    \item
        Для кожної роботи має бути лімітована кількість разів переписування
\end{enumerate}

Тобто, на кожному етапі треба слідкувати за тим, щоб студенти мали справу лише
зі своєю роботою, не мали змоги виправити написану роботу, не могли змінити
результати роботи в журналі викладача, але все це досить тривіальні, на мою
думку, речі, якими з давніх давен займається захист інформації.

Інновації відбуваються на більш абстрактному рівні --- на рівні знань студентів
і співставлення їх з виконаними роботами, що має викладач.
Якщо все ж таки мало місце списування, потрібно це виявити, додати характерні
ознаки ``місця злочину'' до бази знань та на її основі робити подальші висновки
щодо інших студентів.
Є ще цікавий випадок --- коли студенти, що мають знання, перехвилювалися і
наробили помилок.
Звісно, все це аналізувати складно, але спробувати варто --- не можна стояти на
місці.
