\section{Власне, завдання}
Як вже було описано вище, індивідуальне завдання --- створення розподіленої
геоінформаційної системи, призначення якої --- навчання обробці і використанню
даних, що з’являються в таких системах.
Це є векторні, растрові дані, таблиці, текстова інформація, числа, діаграми
і таке інше.
Тобто, дані мають різноманітне походження, потребують застосування різних
методів зберігання та обробки, але треба створити єдину систему, що буде
дозволяти коректно працювати з усіма даними єдиним способом.
Далі ця система буде розширена і узагальнена на інші галузі і інші потреби
--- не тільки навчання студентів, але й навчання та підвищення кваліфікації
персоналу підприємств, і не тільки для використання та обробки геоданих.

\section{Актуальність проблеми}
В світі інформаційних технологій жити стає дуже зручно.
Завдяки потужним комп’ютерам і мережі Інтернет стає можливою передача великих
обсягів інформації за відносно малий час.
Чому б не використати можливості сучасних інформаційних технологій для більшого
покращення процесу навчання?

\section{Аналоги}
Перші кроки вже зроблено --- існують електронні щоденники шкільників,
електронні конспекти з різних дисциплін, електронні тести для перевірки знань.
На мою думку, найбільш розвиненою з таких систем є Coursera, що надає доступ
до навчальних матеріалів (відеолекцій, друкованих конспектів), проводить
перевірку знань у вигляді тестів. Зокрема Coursera має дуже цікаву можливість
підтвердження особистості за допомогою веб-камери та електронного почерку, що
дозволяє видавати сертифікати за проходження курсів, яким можна довіряти.

\section{Прецеденти}

\subsection{Реальне життя}

Послідовність дій студента за курс навчання якогось предмету можна розкласти
на наступні етапи:
\begin{enumerate}
    \item
        Записатися на курс (вступити до університету)
    \item\label{item:learnTheory}
        Ознайомитися з теорією (слухати лекції)
    \item\label{item:doExercises}
        Виконати практичні завдання для закріплення теорії
        (виконання домашніх завдань, контрольних робіт)
    \item
        Повторювати \ref{item:learnTheory} і \ref{item:doExercises} за
        навчальним планом
    \item
        Продемонструвати свої знання та навички у зазначений строк
        (залік, екзамен, курсова)
\end{enumerate}

Послідовність дій викладача:
\begin{enumerate}
    \item
        Стати викладачем
    \item
        Створити план курсу --- план лекцій, практичних занять тощо
    \item
\end{enumerate}
