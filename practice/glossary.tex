%Машинне навчання\newglossaryentry{computer}

\newglossaryentry{bpmn}{
name={BPMN},
description={(англ. Business Process Model and Notation, нотація та
модель бізнес-процесів) система умовних позначень для моделювання
бізнес-процесів}
}

\newglossaryentry{abstraction}{
name={Абстракція},
description={узагальнення більш простих понять до більш
складних, розглядання конкретного явища замість видів, в яких воно може
поставати}
}

\newglossaryentry{machinelearning}{
name={Машинне навчання},
description={підрозділ штучного інтелекту, що вивчає методи побудови моделей,
що здатні до самонавчання}
}

\newglossaryentry{computerstudy}{
name={Комп’ютерне навчання},
description={Навчання людей за допомогою комп’ютера}
}

\newglossaryentry{expertsystem}{
name={Експертна система},
description={Комп’ютерна система, що здатна частково замінити експерта}
}

\newcommand{\textgreek}[1]{\begingroup\fontencoding{LGR}\selectfont#1\endgroup}
\newglossaryentry{ergonomics}{
name={Ергономіка},
description={(від давньогрецького \textgreek{'ergos} --- праця і
\textgreek{n'omos} --- закон) наука, про пристосування робочого місця
(зокрема комп’ютерного) для забезпечення найбільшого комфорту, ефективності
і безпеки}
}

\newglossaryentry{precedence}{
name={Прецедент},
description={специфікація послідовності дій при проектуванні програмних систем}
}

\newglossaryentry{knowledgebase}{
name={База знань},
description={особливого роду база даних, розроблена для управління знаннями
(метаданими), тобто збором, зберіганням, пошуком і видачею знань.
Використовується в експертних системах}
}

\newglossaryentry{framework}{
name={Програмний фреймворк},
description={готовий до використання комплекс програмних рішень, включаючи,
дизайн, логіку та базову функціональність системи або підсистеми}
}

\newglossaryentry{library}{
name={Бібліотека (програмування)},
description={збірка об'єктів чи підпрограм для вирішення близьких за
тематикою задач}
}

\newglossaryentry{inferenceengine}{
name={Машина виведення},
description={програма, яка виконує логічний вивід з попередньо побудованої бази
фактів і правил згідно з законами формальної логіки}
}

\newglossaryentry{classification}{
name={Класифікація},
description={система розподілення об'єктів (процесів, явищ) за класами (групами
тощо) відповідно до визначених ознак}
}

\newglossaryentry{dataclustering}{
name={Кластерний аналіз (кластеризація)},
description={задача розбиття заданої вибірки ситуацій (об'єктів) на підмножини,
що називаються кластерами, так, щоб кожен кластер складався з схожих об'єктів,
а об'єкти різних кластерів істотно відрізнялися. Частковий випадок класифікації}
}

\newglossaryentry{}{
name={},
description={}
}

%\newglossaryentry{}{
%name={},
%description={}
%}
% makeindex index.glo -s index.ist -t index.glg -o index.gls
